\documentclass[pdf]{beamer}
\usepackage[utf8]{inputenc}
\usepackage[T1]{fontenc}
\usepackage{graphicx}
\usepackage{amsmath}
\usepackage{amssymb}
\usepackage{amsfonts}
\usepackage{algorithm,algorithmic}
\bibliographystyle{apacite}

\title{El mundo de la computación y las ciencias de las matemáticas}
\author{Alejandro Cerda y Ignacio Morales}
\date{26 de enero de 2023}
\usebackgroundtemplate{\includegraphics{fondollenobacan.pdf}}
\begin{document}

\maketitle

\section{Índice}


\begin{frame}{Contenido}
    
    \begin{columns}[t]
        \begin{column}{.5\textwidth}
            \tableofcontents[sections={1-5}]
        \end{column}
        \begin{column}{.5\textwidth}
            \tableofcontents[sections={6-11}]
        \end{column}
    \end{columns}
\end{frame}

\begin{frame}{Introducción}
\section{Introducción}

\begin{columns}[t]
        \begin{column}{.5\textwidth}
        
            Parte I

            \begin{footnotesize}
            
            Explicar la nota final de programación
            en función de:

            - Nota final de Estadística I
            
            - Nota final de Métodos Matemáticos III
            
            - Nota final de Contabilidad
            
            - Sexo del estudiante
            
            - Residencia (Vive o no en Stgo.)

            \end{footnotesize}
            
        \end{column}
        \begin{column}{.5\textwidth}
        
            Parte II

            \begin{footnotesize}
            
            Explicar el promedio de los ramos de Estadística
            en función de:

            - Carrera
            
            - Sexo
            
            - Si juega videojuegos multijugador online
            
            Si juega videojuegos, entonces se añade:

            - Horas de juego

            - Horas que mira stream/videos de juegos

            - Tipo de contenido que mira

            - Rango máximo alcanzado en Ranked
            
            \end{footnotesize}
            
        \end{column}
\end{columns}


\end{frame}
\begin{frame}{Parte I}
\section{Parte I}
\begin{footnotesize} 
La siguiente tabla muestra las estadísticas de los 4 ramos a considerar:

\begin{table}[ht]
\centering
\begin{tabular}{rllllll}
  \hline
  Programación & Estadística & Matemática & Contabilidad \\ 
  \hline
  Min.   :1.000   & Min.   :1.798   & Min.   :2.000   & Min.   :2.964  \\ 
  1st Qu.:4.100   & 1st Qu.:3.570   & 1st Qu.:5.175   & 1st Qu.:4.251  \\ 
  Median :4.709   & Median :4.160   & Median :5.900   & Median :4.686  \\ 
  Mean   :4.630   & Mean   :4.121   & Mean   :5.624   & Mean   :4.754  \\ 
  3rd Qu.:5.341   & 3rd Qu.:4.673   & 3rd Qu.:6.325   & 3rd Qu.:5.072  \\ 
  Max.   :6.900   & Max.   :6.279   & Max.   :6.800   & Max.   :6.254  \\ 
   \hline
\end{tabular}
\end{table}

Y el siguiente gráfico, la densidad de notas por ramo:
\begin{center}
\includegraphics[trim={1cm 1cm 1cm 1cm},width=0.6\linewidth]{rplot01.pdf}
\end{center}
Además, el 59\% de los alumnos son hombres y el 71\% vive en Stgo.

\end{footnotesize}

\end{frame}
\begin{frame}{Parte I}

\begin{center}
\begin{table}[ht]
\centering
\begin{tabular}{rrrrr}
  \hline
 & Estimate & Std. Error & t value & Pr($>$$|$t$|$) \\ 
  \hline
(Intercept) & 1.4236 & 0.9348 & 1.52 & 0.1323 \\ 
  notas\_estad & 0.2378 & 0.1496 & 1.59 & 0.1166 \\ 
  notas\_mateIII & 0.3394 & 0.1627 & 2.09 & 0.0407 \\ 
  notas\_conta & 0.1222 & 0.2123 & 0.58 & 0.5667 \\ 
  sexo\_var & -0.4921 & 0.4406 & -1.12 & 0.2679 \\ 
  r\_santiago & -0.2032 & 0.3980 & -0.51 & 0.6112 \\ 
  sexo\_var:r\_santiago & 0.4234 & 0.5108 & 0.83 & 0.4100 \\ 
   \hline
\end{tabular}
\end{table}
\tiny{Multiple R-squared: 0.3046, Adjusted R-squared: 0.2442.
F-statistic: 5.038 on 6 and 69 DF, p-value: 0.0002462}
\end{center}

\end{frame}

\begin{frame}{Parte I Modelo Final}
\section{Parte I Modelo Final}

\begin{center}
\begin{table}[ht]
\begin{center}
\begin{tabular}{rrrrr}
  \hline
 & Estimate & Std. Error & t value & Pr($>$$|$t$|$) \\ 
  \hline
(Intercept) & 1.3083 & 0.6405 & 2.04 & 0.0447 \\ 
  notas\_estad & 0.2521 & 0.1426 & 1.77 & 0.0813 \\ 
  notas\_mateIII & 0.4060 & 0.1431 & 2.84 & 0.0059 \\ 
   \hline
\end{tabular}
\end{center}
\end{table}

\tiny{Multiple R-squared: 0.2837, Adjusted R-squared: 0.2641.
F-statistic: 14.46 on 2 and 73 DF, p-value: 5.139e-06}
\end{center}

\begin{center}

\[Pr_i = 1.3083 + 0.2521 E_i + 0.4060 M_i + \epsilon_i\]
\end{center}

\(Pr_i\) = Nota final de programación

\(E_i\) = Nota final de estadística

\(M_i\) = Nota final de matemática

\(R^2\) ajustado de nuestro modelo final: 0.2641

\end{frame}
\begin{frame}{Parte I Conclusión}
\section{Parte I Conclusión}
 \begin{footnotesize}
            
            Las variables más eficaces para la predicción de la nota final de programación, son:

            - Nota final de Estadística I
            
            - Nota final de Métodos Matemáticos III

            Dentro de las variables que consideramos para nuestro modelo
            
            

            \end{footnotesize}
            \nocite{comuna}
            \nocite{estrescovid}

\end{frame}
\begin{frame}{Parte II}
Estadísticas de los encuestados:


\includegraphics[trim={1.5cm 0.5cm 1cm 0},width=0.45\linewidth,scale=1]{carreras.pdf}
\includegraphics[trim={1.5cm 0.5cm 1cm 0},width=0.45\linewidth,scale=1]{sexo.pdf}

\includegraphics[trim={0.25cm 1cm 0.25cm 1cm},width=\linewidth]{estadistica1_2.pdf}
\section{Parte II}

\end{frame}
\begin{frame}{Parte II}
Horas de juego y stream de los encuestados:
\vspace{.4cm}

\includegraphics[trim={cm 0cm 0cm 0},width=0.45\linewidth,scale=1]{hjuego.pdf}
\includegraphics[trim={cm 0cm 0cm 0},width=0.45\linewidth,scale=1]{hstream.pdf}

Elo relativo de los encuestados:

\begin{center}
\includegraphics[trim={0.25cm cm 0.25cm cm},width=0.8\linewidth]{elo2.pdf}
\end{center}

\end{frame}
\begin{frame}

\begin{center}
\begin{tiny}
    

\begin{table}[ht]
\centering
\begin{tabular}{rrrrr}
 \hline
& Estimate & Std. Error & z value & Pr($>$$|$z$|$) \\
 \hline
(Intercept)                                               &  6.8362  &    0.6949 &  9.84 & < 2e-16\\
sexo\_var &                                                   0.3033  &    0.2021  & 1.50  &0.13346\\
carreraExterno  &                                          0.3930      &0.2704   &1.45  &0.14607\\
carreraIC MA    &                                        0.4224   &   0.2904  & 1.45  &0.14583\\
carreraIC\ ME     &                                        0.0736     & 0.2856  & 0.26  &0.79663\\
carreraIICG        &                                        0.2709    &  0.2721  & 1.00 & 0.31947\\
horas\_juegoEntre 15 y 20       &                            -0.1047   &   0.4376  &-0.24 & 0.81089\\
horas\_juegoEntre 20 y 25     &                              -0.4720  &    0.4158 & -1.14 & 0.25623\\
horas\_juegoEntre 5 y 10       &                             -0.0707     & 0.3079 & -0.23  &0.81849\\
horas\_juegoMás de 25                    &                    0.5369    &  0.5424  & 0.99 & 0.32224\\
horas\_juegoMenos de 5             &                         -0.1765   &   0.3237 & -0.55 & 0.58554\\
horas\_juegoNo juega online   &                              -2.4048    &  0.6858  &-3.51 & 0.00045\\
horas\_streamEntre 15 y 20    &                              -1.1203 &     0.5164  &-2.17 & 0.03007\\
horas\_streamEntre 5 y 10      &                             -0.7635   &   0.3624  &-2.11 & 0.03514\\
horas\_streamMás de 25            &                          -0.6368 &     0.9090  &-0.70 & 0.48359\\
horas\_streamMenos de 5                  &                   -1.3738  &    0.3411 & -4.03 & 5.6e-05\\
contenido para mejorar en el juego &                         -0.5187  &    0.6071  &-0.85 & 0.39293\\
contenido por diversión               &                      -0.8006    &  0.3983  &-2.01 & 0.04444\\
contenido un mix de ambas                &                   -0.7533   &   0.4287 &-1.76 & 0.07890\\
elo\_maximo Elo alto                  &                       -0.6342    & 0.4508  &-1.41  &0.15946\\
elo\_maximo Elo bajo              &                           -0.8491  &    0.5692  &-1.49 & 0.13579\\
elo\_maximo Elo medio          &                              -0.8730    &  0.4294  &-2.03 & 0.04204\\
elo\_maximo Elo medio alto                  &                 -0.3039   &   0.4329 & -0.70 & 0.48267\\
elo\_maximo Elo medio bajo          &                         -0.7331   &   0.5339  &-1.37  &0.16971\\
elo\_maximo Introductorio               &                     -0.5534    &  0.9413  &-0.59  &0.55657\\
elo\_maximoNunca he jugado ranked        &                   -0.7847     & 0.5329  &-1.47  &0.14088\\
    \hline
   \end{tabular}
   \end{table}
\tiny{Gaussian distribution. Scale= 0.732, Loglik(model)= -191.1 . Chisq= 32.37 on 27 degrees of freedom, p= 0.22  Number of Newton-Raphson Iterations: 4}
\end{tiny}
\end{center} 


\end{frame}
\begin{frame}{Parte II Modelo Final}

\begin{center}
\begin{small}
\begin{table}[ht]
\centering
\begin{tabular}{rrrrr}
 \hline
& Estimate & Std. Error & z value & Pr($>$$|$z$|$) \\
 \hline
(Intercept)                & 5.5474    & 0.3429& 16.18& <2e-16\\
sexo\_var                  & 0.3197 &    0.1809  & 1.77& 0.0772\\
horas\_stream entre 15 y 20 &-1.2425    & 0.5077 &-2.45& 0.0144\\
horas\_stream entre 5 y 10 & -0.6889   &  0.3519 &-1.96& 0.0503\\
horas\_stream más de 25   &  -1.1171    & 0.8656& -1.29& 0.1968\\
horas\_stream menos de 5   & -1.0742    & 0.3386& -3.17& 0.0015\\
horas\_stream no consume   & -0.8701    & 0.3467 &-2.51& 0.0121\\
    \hline
   \end{tabular}
   \end{table}
\tiny{Gaussian distribution. Scale=0.796 , Loglik(model)= -199.5. Chisq= 15.49 on 6 degrees of freedom, p= 0.017. AIC=415.0184 , BIC=436.2501 }
\end{small}
\end{center}   

\begin{center}
\begin{tiny}
\math{Es_i = 5.547 + 0.320Hom_i - 1.243E15y30_i - 0.689E5y10_i - 1.117M25_i - 1.074M5_i - 0.870NC_i + \epsilon_i}
\end{tiny}
\end{center}

\begin{small}

\(Es_i\) = Nota final de estadística

\(E15y30_i , E5y10_i , M25_i, M5_i , NC_i\) = Cantidad de horas de stream que consume (V. Categórica)

\(Hom_i\) = dummy que toma el valor 1 si el alumno es hombre

\(AIC\) de nuestro modelo final: 415.0184

\end{small}

\section{Parte II Modelo Final}

\end{frame}
\begin{frame}{Parte II Conclusión}
\section{Parte II Conclusión}
\begin{footnotesize}
            
Las variables más utiles para predecir la nota final, dentro de las consideradas, son:

- Horas que el alumno dedica a ver contenido relacionado
            
- Sexo del estudiante

            
Supuestos a considerar:

- Supuesto de distribución normal

- Gente que no juega, no ve contenido de esos juegos

- Estudiantes mantienen su juego, consumo y elo durante el tiempo

            \end{footnotesize}


\end{frame}
\begin{frame}{Conclusión final}
\section{Conclusión final}

- Importancia de ramos matemáticos frente a las nuevas tecnologías

- Nuevas variables, nuevas explicaciones


\end{frame}

\begin{frame}{Referencias}
\nocite{women}
\begin{tiny}
\bibliography{referencias.bib}
\end{tiny}
\section{Referencias}
\end{frame}


\end{document}


